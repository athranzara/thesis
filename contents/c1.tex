

\chapter{Introduction}

In response to rapid traffic growth in Datacenter through Ethernet links, there has been a rising interest to apply optical  techniques to Ethernet \ac{phy}. Among all \ac{mimo} solutions, \ac{mdm}, with its capability to operate over conventional \ac{gi-mmf}, is proposed as one of potential solutions for \ac{mimo} Ethernet. The introduction of the \ac{slm} accelerates such research, since it is possible to selectively launch spatial and polarized modes within fibre with reasonable isolation\cite{6227317}. Since then, notable work has been demonstrated to improve channel quality \cite{6706113,shi2015sparse}. It could be observed from these experimental demonstrations that the channel is well constrained so that nonlinearity would be minimized. However, crosstalk, \ac{dmd}, \ac{mdl} are still important factors in channel quality degradation. Tackling these problems in \ac{mdm} Ethernet in the optical domain is challenging, since the design for Ethernet links require flexibility and scalability. As a result, \ac{dsp} is commonly used alongside with optical solutions. These solutions will generally be implemented as an ASIC between the optical channel and Ethernet PHY. Though offering great channel quality improvement, they are not adaptive and largely depends on the \ac{mdm} method that is applied. The introduction of NetFPGA-SUME\cite{6866035} enables PHY implementation and research which brings the possibility of merging adaptive \ac{dsp} with Ethernet PHY.


In this paper, we demonstrate an adaptive solution to compensate for crosstalk and \ac{dmd} in Ethernet PHY. Such proposal modified the \ac{pcs} to offer delay tolerance and implement a burst-targeting \ac{fec} to handle crosstalk. The modification in the physical layer is completely transparent to the higher layer, \textit{i.e.} fully compatible when interfacing with MAC. In order to study the scalability for high-port count MDM Ethernet, we implemented an equivalent channel model within the proposed physical layer, in which it is potentially possible to scale up to 10 channels.


%photonic lanterns\cite{Birks:15}

\section{Post-scheduled PCS with FEC compensation}

We build a 2x2 MDM channel with the same setup demonstrated in\cite{6706113}, in which the signal source is replaced with a NetFPGA-SUME board. There XGMII\footnote{A standard interface between PHY and MAC} frames are generated, identical to Ethernet Layer 2 frames generated by MAC. The frame propagates through our proposed PHY and then is transmitted through 10GBASE EML SFP+ transceiver (@1550nm). Our \ac{pcs} architecture design is based on 10GBASE-R \ac{pcs} which is defined in IEEE clause 49. An overview of the physical layer design is demonstrated in Fig.~\ref{fig:phy}.

First, we demonstrate the mechanism of DMD compensation. Compare to original \ac{pcs}, an external FIFO \textless31:0\textgreater is attached to the receive queue of the XGMII interface. It is measured that the DMD for LP$_{1,2}$ mode is 0.61 ns/km \cite{shi2015slm}. For our case, a 2km link yields 12 symbols at line rate of 10.3125 Gb/s. However, for the serialization in \ac{pma}, the arrive first symbol in \ac{pcs} layer varies from 12 bits to 78 bits since, after block encoder the bus toward GTH transceiver is 66 bits; at least 3 clock cycles after the block decoder. The latency scales twice at XGMII since its half the bus width. Considering latching latency of one clock cycle in sub module (block synchronizer, descramble, block decoder, CRC8 remover LDPC decoder), this scales to up to 8 clocks. In our implementation, a depth of 15 fifo \textless31:0\textgreater is implemented, so that it is sufficient latency compensation. As an indication, control code of 0x37\footnote{The control codes are unued XGMII control codes assigned with a hamming distance of 2 to existing control codes} is inserted to indicate the data in current XGMII lane is arrived. This will be used for lane synchronization as part of post processing. 

\begin{figure*}[h]%[htdp]
   \centering
   \includegraphics[width=0.9\textwidth]{phy}
  \caption{Physical layer architecture (yellow blocks are implemented logics, blue blocks are ASICs implemented inside FPGA): a) standard Ethernet physical layer architecture for 10GBASE-KR, an optional FEC is offered. b) Proposed physical layer for a single lane. c) multi-lane physical layer post process}
    \label{fig:phy}
\end{figure*}

We now examine the crosstalk in our system. The crosstalk in our experiments is generated from two sources: MUM/DEMUX and coupling in the fibre. Shown in Table~\ref{tab:xtalk}, MUX/DEMUX is demonstrated to introduce significantly more crosstalk. Previous work demonstrated that, it is measured in such short haul system, crosstalk is significantly stronger than other nonlinearity introduced channel degradation\cite{shi2015slm}. Hence, we can assume optical power constrained in fibre did not enter nonlinear regime. Note that these measurements are optical channel measurements, FEC compensation are not considered yet.

\begin{table}[h]
\centering
\caption{Crosstalk measurements}
\label{tab:xtalk}
\footnotesize
\begin{tabular}{|c|c|c|c|c|c|}
\hline
\multicolumn{3}{|c|}{MUX/DEMUX Crosstalk (dB)} & \multicolumn{3}{c|}{Coupling Crosstalk (dB)} \\ \hline
               & LP$_{0,1}$    & LP$_{1,2}$    &               & LP$_{0,1}$    & LP$_{1,2}$   \\ \hline
LP$_{0,1}$     & -             & -14.9         & LP$_{0,1}$    & -             & -28.2        \\ \hline
LP$_{1,2}$     & -13.7         & -             & LP$_{1,2}$    & -22.3         & -            \\ \hline
\end{tabular}
\end{table}

Finally, we introduce the FEC implementation in our system. The \ac{phy} \ac{fec} specified in 10GBASE-KR architecture offers an optional cyclic code sub-layer implemented between \ac{pcs} and \ac{pma}. Such binary code recycles the synchronization bits overhead that is used in 64B/66B block codecs so that no extra overhead is introduced during ECC encoding. Though targeting burst error, such cyclic code could only offer very limited burst error correction\cite{szczepanek200610gbase}. As a reference, we use Xilinx 10GBASE-KR \ac{ip} core as study case. We estimate the extra logic by compare 10GBASE-KR with 10GBASE-R. An extra 3450 LUTs and 3540 Flip-Flops are observed due FEC implementation while \ac{pcs} end-to-end loopback test indicates a total 384 ns ($\pm 6.4$ ns)\footnote{The core is running under 156.25 MHz, hence introduce a time stamp resolution of 6.4 ns} extra latency.

In our \ac{fec} design, we make a trade-off among burst error coverage, latency and hardware complexity. To overcome burst error, \ac{rs} is considered for its record of high burst error tolerance. However, since \ac{rs} codecs are generally implemented in non-binary field (eg. Galois Field (8)), multiple flip-flop stages are required in hardware design. This directly leads to higher latency. In the other hand, \ac{ldpc} with interleaving is also reported to offer very good burst error tolerance. Although it suffers issues in hardware complexity when applies large parity check matrix and soft-decision decoding, we can overcome these design obstacles by reducing parity check matrix size and using hard-decision. In this demonstration, we implement a regular binary LDPC parity check matrix (3,8)(512,430) with a bit-flipping hard decision decoder. Meanwhile, a GF(8) \ac{rs} is also implemented for comparison. In order to match the same overhead ratio with the proposed \ac{ldpc}, the message block size of \ac{rs} is designed to be 215.

Our design is targeted to consume similar or even less FPGA resource compare to 10GBASE-KR backplane while achieving higher coding gain. Due to the fact that it is build between block encoder/decoder and XGMII interface, the coding ratio is increased to map the XGMII interface. The hardware implementation shows and extra 2970 LUTs and 3120 Flip-Flops are consumed due to our \ac{ldpc} and XGMII logic design. The \ac{pcs} end-to-end loopback test indicates a total extra latency of only 230.4 ns ($\pm 6.4$ ns). 

\begin{figure}[h]
  \centering
  \includegraphics[width=0.47\textwidth]{mdm_pulse_distribution}
  \caption{Flat Gaussian Pulse invoked into primary mode of mode division multiplexed multimode fibre}
  \label{fig:mdm_pulse}
\end{figure}


\section{Embedded Channel Model}
The coding gain evaluation requires to adjust and monitor SNR in the fibre channel. In order to reduce research cycle, we implemented an adaptive embedded channel model that is directly attached to the GTH transceiver. In a system perspective, such attachment is equivalent as attaching to the end of PMA layer. Our emulation model describes the optical fields in MDM channels which is demonstrated in Eq.~\ref{eq:field}. The previous measurements indicates the optical power operates within linear regime, hence we can ignore the second term in the equation.
\begin{eqnarray}
\frac{\partial U(t,z))}{\partial z} &{}={}& - j\frac{\beta_2}{2} \frac{\partial^2U(t,z)}{\partial t^2} \nonumber\\
                                    && + j\gamma |U(t,z)|^2 U(t,z) \nonumber \\
                                    && -\frac{\alpha}{2} U (t,z) + W(t,z)
\label{eq:field}
\end{eqnarray}
We insert loss (considering both attenuation loss and MDL) with the term $-\alpha/2 U (t,z)$. In addition, we consider noises as a waveform function $W(t,z)$. The noise in MDM systems is mainly contributed two attributes: 1) thermal noise that is generated electronic devices and feed into optical channels and 2) cross-talk from other MDM channels. The first source of noise can be considered as AWGN. The second source of noise is not as AWGN since it is highly dependent on the data pattern of other MDM channels. However, the proposed MDM channel is targeting Ethernet architecture. After processing through Ethernet physical layer, the data is white noise\cite{audzevich2013low}. Thus, we simplify such crosstalk as AWGN. A visual representation of a Gaussian pulse model propagating through MDM channel is shown in Fig.~\ref{fig:mdm_pulse}. The hardware representation utilizing accumulated Box Muller method\cite{911557}.

\begin{table}[h]
\centering
\caption{Summary of linear distortion parameters}
\label{tab:sum}
\footnotesize
\begin{tabular}{|l|l|l|}
\hline
                                  & LP$_{0,1}$    & LP$_{1,2}$   \\ \hline
wavelength ($\lambda$ nm)         & \multicolumn{2}{l|}{1550} \\ \hline
core index (n$_1$)                & \multicolumn{2}{l|}{1.485}   \\ \hline
cladding index (n$_2$)            & \multicolumn{2}{l|}{1.477}   \\ \hline
core radius ($\mu$m)              & \multicolumn{2}{l|}{50}      \\ \hline
A$_{eff}$ ($\mu$)m$^2$            & 101.55        & 137.1        \\ \hline
Chromatic Dispersion ($ps/km/nm$) & 24.2          & 21.8         \\ \hline
Dispersion Slope ($ps/nm^2/km$)   & 0.0663        & 0.0658       \\ \hline
PMD ($ps/\sqrt{km}$)              & \multicolumn{2}{l|}{0.137}   \\ \hline
MDL  (dB/km)                      & 0.278         & 0.494        \\ \hline
\end{tabular}
\end{table}

\section{Results}
Fig. \ref{fig:ecc} shows the measurement of normalized coding gain performance for \ac{rs} and \ac{ldpc} over LP$_{0,1}$ and LP$_{1,2}$ channel. For both channel, the modified LDPC with proposed PCS architecture offers better performance, a coding gain of 5.2 dB for LP$_{0,1}$ and 4.2 dB for LP$_{1,2}$ channel correspondingly. An approximately 1 dB coding gain is observed compare to equivalent \ac{rs} coder. Considering the crosstalk and \ac{mdl} requirement is above 15 dB, the proposed architecture satisfy the design requirement.
Note that the FEC choice in our demonstration is a regular LDPC with relatively small parity check matrix and a hard decision decoder. However, we demonstrated that with our proposed \ac{pcs} design, a 2x2 MDM channel over Ethernet is still possible with such minimal effort of FEC. 


\begin{figure}[h]
  \centering
  \includegraphics[width=0.45\textwidth]{lpchannel_c}
  \caption{BER vs EB/No comparison for Reed Solomon and LDPC under LP(0,1) channel and LP(1,2) channel}
  \label{fig:ecc}
\end{figure}



%channel error generator embedded in PHY
%introduce physical layer architecture here

%\section{Physical Layer Architecture}


\section{Conclusions}

We demonstrated an adaptive \ac{phy} for short-haul 2x2 MIMO using mode division multiplexing that targets Ethernet application. It is transparent and fully compatible with higher layer. With its multi-lane architecture it offers a potential scalability. By using PCS backplane FEC processing, the architecture provides good FEC flexibility. In our experimental demonstration, we show that it could enable 2x2 MDM Ethernet Links with minimal latency cost and FPGA resources.

 


\section{Acknowledgements}
The authors would like to acknowledge Adrian Wonfor for his insight in system design.

%-------------------------------------------------- Section 9 -------------------------------------------------------%

%\section{Acknowledgements}
%We wish to thank suggestion and support from reviewers.

%-------------------------------------------------- Literature -------------------------------------------------------%
\bibliographystyle{abbrv}
\bibliography{ecoc2017}

%\bibliographystyle{abbrv}
%\begin{spacing}{1.35}

%\begin{thebibliography}{1}
%\bibitem{ref1}
%F. M. Lastname et al., ``\uline{Full Paper Titles are Mandatory for Adequate Referencing in Web Search Engines},'' J. Lightwave Technol., Vol. {\bf 12}, no. 5, p. 456 (1990).
%\bibitem{ref2}
%F. M. Lastname et al., ``Why Should we Use Photonics?,'' Proc. OFC, WeA3, Anaheim (2005).
%\bibitem{ref3}
%F. M. Lastname et al., ``Connecting the Dots,'' Proc. ECOC, Tu.9.G.1, London (2005).
%\bibitem{ref4}
%F. M. Lastname et al., ``Why Connect the Dots?,'' Photon. Technol. Lett., Vol. {\bf 12}, no. 3, p. 34 (2001).
%\bibitem{ref5}
%F. M. Lastname et al., ``The Dots are Connected,'' Proc. OFC, PDP41, San Diego (2010).
%\bibitem{ref6}
%F. M. Lastname, How to Write a Paper, Publishing Company (2002). 
%\end{thebibliography}
%\end{spacing}
%\vspace{-4mm}

%%%%%%%%%%%%%%%%%%%%%%%%%%%%%%%%%%%%%%%%%%%%%
%---------------------------------------------- End of Document -----------------------------------------------%

